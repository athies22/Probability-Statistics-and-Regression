%!TEX output_directory = temp
\documentclass{amsart}
	%% Basic Info
		\author{Alex Thies}
		\title{Homework 6 \\ Math 463 - Spring 2017}
		\email{athies@uoregon.edu}
	%% boilerplate packages
		\usepackage{amsmath}
		\usepackage{amssymb}
		\usepackage{amsthm}
		\usepackage{enumerate}
		\usepackage{mathrsfs}
		\usepackage{color}
		\usepackage{hyperref}
	%% rename the abstract
		\renewcommand{\abstractname}{Introduction}
	%% my shorthand
	    % sets
	    \DeclareMathOperator{\Z}{\mathbb{Z}}
	    \DeclareMathOperator{\Zp}{\mathbb{Z^{+}}}
        \DeclareMathOperator{\Zm}{\mathbb{Z^{-}}}
        \DeclareMathOperator{\N}{\mathbb{Z^{+}}}
	    % linear algebra stuff
		\DeclareMathOperator{\Ell}{\mathscr{L}}
		% stats stuff
		\DeclareMathOperator{\var}{\rm Var}
		\DeclareMathOperator{\sd}{\rm SD}
		\DeclareMathOperator{\SE}{\rm SE}
		% use pretty characters
		\DeclareMathOperator{\ep}{\varepsilon}
		\DeclareMathOperator{\ph}{\varphi}
	%% Levin's shorthand
		\newcommand{\E}{{\mathcal{E}}}
		\newcommand{\A}{{\mathcal{A}}}
		\newcommand{\B}{{\mathcal{B}}}
		\newcommand{\R}{{\mathbb{R}}}
		\newcommand{\X}{{\mathbf{X}}}
		\newcommand{\x}{{\mathbf{x}}}
		\newcommand{\M}{{\mathcal{M}}}
		\newcommand{\bvec}[1]{{\boldsymbol #1}}
		\newcommand{\bbeta}{\bvec{\beta}}
		\newcommand{\bX}{\bvec{X}}
		\newcommand{\bY}{\bvec{Y}}
		\newcommand{\ssreg}{{\rm SS}_{{\rm Reg}}}
		\newcommand{\ssr}{{\rm SS}_{{\rm Res}}}
		\newcommand{\sst}{{\rm SS}_{{\rm Tot}}}
\begin{document}
	% \begin{abstract}
	% 	I collaborated with Joel Bazzle, Torin Brown, Andy Heeszel, Ashley Ordway, and Seth Temple on this assignment.
	% \end{abstract}

	\maketitle

	\section{Assignment} % (fold)
	\label{sec:assignment}
		\subsection{Problem 7.A.12} % (fold)
		\label{sub:problem_7_a_12}
			Suppose $X$, $Y$, $Z$ are independent normal random variables, each having variance 1. 
			The means are $\alpha + \beta$, $\alpha + 2\beta$, $2\alpha + \beta$, respectively: $\alpha$, $\beta$ are parameters to be estimated. 
			Show that maximum likelihood and OLS give the same estimates. 
			Note: this won’t usually be true—the result depends on the normality assumption.
		\begin{proof}[Solution]
		\end{proof}
		% subsection problem_7_a_12 (end)

		\subsection{Problem 7.D.2} % (fold)
		\label{sub:problem_7_d_2}
			Conversely, if $U$ is uniform on $[0,1]$, show that $F^{-1}(U)$ has distribution function $F$.
			(This idea is often used to simulate IID picks from $F$.)
		\begin{proof}[Solution]
		\end{proof}
		% subsection problem_7_d_2 (end)

		\subsection{Problem 7.D.7} % (fold)
		\label{sub:problem_7_d_7}
			\begin{enumerate}[(a)]
				\item If $\mathbb{P}(Y_{i} = 1|X) = \Lambda(X_{i}\beta)$, show that logit $\mathbb{P}(Y_{i} = 1|X) = X_{i}\beta$.
				\item If logit $\mathbb{P}(Y_{i} = 1|X) = X_{i}\beta$, show that $\mathbb{P}(Y_{i} = 1|X) = \Lambda(X_{i}\beta)$.
			\end{enumerate}
		\begin{proof}[Solution] \
			\begin{enumerate}[(a)]
				\item 
				\item 
			\end{enumerate}
		\end{proof}
		% subsection problem_7_d_7 (end)

		\subsection{Problem 7.D.13} % (fold)
		\label{sub:problem_7_d_13}
			Show that the log likelihood for the probit model is concave, and strictly concave if $X$ has full rank. 
			Hint: this is like exercise 11.
		\begin{proof}[Solution]
		\end{proof}
		% subsection problem_7_d_13 (end)

		\subsection{Problem 7.E.6} % (fold)
		\label{sub:problem_7_e_6}
			Student \# 77 is Presbyterian, went to public school, and graduated. 
			What does this subject contribute to the likelihood function? 
			Write your answer using $\ph$ in equation (15).
		\begin{proof}[Solution]
		\end{proof}
		% subsection problem_7_e_6 (end)

		\subsection{Problem 7.6} % (fold)
		\label{sub:problem_7_6}
			Powers and Rock (1999) consider a two-equation model for the effect of coaching on SAT scores:
				\begin{align*}
					X_{i} &= 1 \text{ if } \it U_{i}\alpha + \delta_{i} > 0\text{, else }X_{i} = 0 & \rm (assignment) \\
					Y_{i} &= cX_{i} + V_{i}\beta + \sigma \ep_{i} & \rm (response)
				\end{align*}
			Here, $X_{i} = 1$ if subject $i$ is coached, else $X_{i} = 0$. 
			The response variable $Y_{i}$ is subject $i$'s SAT score; $U_{i}$ and $V_{i}$ are vectors of personal characteristics for subject $i$, treated as data. 
			The latent variables $(\delta_{i}, \ep_{i})$ are I.I.D. bivariate normal with mean 0, variance 1, and correlation $\rho$; they are independent of the $U$'s and $V$'s. 
			(In this problem, $U$ and $V$ are observable, $\delta$ and $\ep$ are latent.)
			\begin{enumerate}
				\item Which parameter measures the effect of coaching? 
				How would you estimate it?
				\item State the assumptions carefully (including a response schedule, if one is needed). 
				Do you find the assumptions plausible?
				\item Why do Powers and Rock need two equations, and why do they need $\rho$?
				\item Why can they assume that the disturbance terms have variance 1? 
				Hint: look at sections 7.2 and 7.4.
			\end{enumerate}
		\begin{proof}[Solution] \
			\begin{enumerate}[(a)]
				\item
				\item 
				\item 
				\item 
			\end{enumerate}
		\end{proof}
		% subsection problem_7_6 (end)

		\subsection{Problem 7.7} % (fold)
		\label{sub:problem_7_7}
			Shaw (1999) uses a regression model to study the effect of TV ads and candidate appearances on votes in the presidential elections of 1988, 1992, and 1996. 
			With three elections and 51 states (DC counts for this purpose), there are 153 data points, i.e., pairs of years and states. 
			Each variable in the model is determined at all 153 points. In a given year and state, the volume TV of television ads is measured in 100s of GRPs (gross rating points). 
			Rep.TV , for example, is the volume of TV ads placed by the Republicans. 
			AP is the number of campaign appearances by a presidential candidate. 
			UN is the percent undecided according to tracking polls. 
			PE is Perot’s support, also from tracking polls. 
			(Ross Perot was a maverick candidate.) 
			RS is the historical average Republican share of the vote. 
			There is a dummy variable D1992, which is 1 in 1992 and 0 in the other years. 
			There is another dummy D1996 for 1996. 
			A regression equation is fitted by OLS, and the Republican share of the vote is
			\begin{align*}
				&\rm- 0.326 - 2.324\times D_{1992} - 5.001\times D_{1996} \\
				&\rm + 0.430 \times (Rep. TV - Dem. TV ) + 0.766\times (Rep. AP - Dem. AP ) \\
				&\rm + 0.066 \times (Rep. TV - Dem. TV ) \times (Rep. AP - Dem. AP ) \\
				&\rm + 0.032 \times (Rep. TV - Dem. TV ) \times UN + 0.089 \times (Rep. AP - Dem. AP )\times UN \\
				&\rm + 0.006 \times (Rep. TV - Dem. TV ) \times RS + 0.017 \times (Rep. AP - Dem. AP )\times RS \\
				&\rm + 0.009 \times UN + 0.002 \times PE + 0.014 \times RS + error.
			\end{align*}
			\begin{enumerate}[(a)]
				\item What are dummy variables, and why might D1992 be included in the equation?
				\item According to the model, if the Republicans buy another 500 GRPs in a state, other things being equal, will that increase their share of the vote in that state by $0.430 \times 5 \stackrel{.}{=} 2.2$ percentage points?
				Answer yes or no, and discuss briefly. 
				(The $0.430$ is the coefficient of Rep. TV − Dem. TV in the second line of the equation.)
			\end{enumerate}
		\begin{proof}[Solution] \
			\begin{enumerate}[(a)]
				\item 
				\item 
				\item 
			\end{enumerate}
		\end{proof}
		% subsection problem_7_7 (end)
	% section assignment (end)
\end{document}