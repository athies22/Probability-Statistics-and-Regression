\documentclass{amsart}\usepackage[]{graphicx}\usepackage[]{color}
%% maxwidth is the original width if it is less than linewidth
%% otherwise use linewidth (to make sure the graphics do not exceed the margin)
\makeatletter
\def\maxwidth{ %
  \ifdim\Gin@nat@width>\linewidth
    \linewidth
  \else
    \Gin@nat@width
  \fi
}
\makeatother

\definecolor{fgcolor}{rgb}{0.345, 0.345, 0.345}
\newcommand{\hlnum}[1]{\textcolor[rgb]{0.686,0.059,0.569}{#1}}%
\newcommand{\hlstr}[1]{\textcolor[rgb]{0.192,0.494,0.8}{#1}}%
\newcommand{\hlcom}[1]{\textcolor[rgb]{0.678,0.584,0.686}{\textit{#1}}}%
\newcommand{\hlopt}[1]{\textcolor[rgb]{0,0,0}{#1}}%
\newcommand{\hlstd}[1]{\textcolor[rgb]{0.345,0.345,0.345}{#1}}%
\newcommand{\hlkwa}[1]{\textcolor[rgb]{0.161,0.373,0.58}{\textbf{#1}}}%
\newcommand{\hlkwb}[1]{\textcolor[rgb]{0.69,0.353,0.396}{#1}}%
\newcommand{\hlkwc}[1]{\textcolor[rgb]{0.333,0.667,0.333}{#1}}%
\newcommand{\hlkwd}[1]{\textcolor[rgb]{0.737,0.353,0.396}{\textbf{#1}}}%
\let\hlipl\hlkwb

\usepackage{framed}
\makeatletter
\newenvironment{kframe}{%
 \def\at@end@of@kframe{}%
 \ifinner\ifhmode%
  \def\at@end@of@kframe{\end{minipage}}%
  \begin{minipage}{\columnwidth}%
 \fi\fi%
 \def\FrameCommand##1{\hskip\@totalleftmargin \hskip-\fboxsep
 \colorbox{shadecolor}{##1}\hskip-\fboxsep
     % There is no \\@totalrightmargin, so:
     \hskip-\linewidth \hskip-\@totalleftmargin \hskip\columnwidth}%
 \MakeFramed {\advance\hsize-\width
   \@totalleftmargin\z@ \linewidth\hsize
   \@setminipage}}%
 {\par\unskip\endMakeFramed%
 \at@end@of@kframe}
\makeatother

\definecolor{shadecolor}{rgb}{.97, .97, .97}
\definecolor{messagecolor}{rgb}{0, 0, 0}
\definecolor{warningcolor}{rgb}{1, 0, 1}
\definecolor{errorcolor}{rgb}{1, 0, 0}
\newenvironment{knitrout}{}{} % an empty environment to be redefined in TeX

\usepackage{alltt}
	%% Basic Info
		\author{Alex Thies}
		\title{Homework 4 \\ Math 463 - Spring 2017}
		\email{athies@uoregon.edu}
	%% boilerplate packages
		\usepackage{amsmath}
		\usepackage{amssymb}
		\usepackage{amsthm}
		\usepackage{enumerate}
		\usepackage{mathrsfs}
		\usepackage{color}
		\usepackage{hyperref}
	%% rename the abstract
		\renewcommand{\abstractname}{Introduction}
	%% graphics stuff
		\usepackage{graphicx}
		\graphicspath{{../images/}}
	%% my shorthand
	    % sets
	    \DeclareMathOperator{\Z}{\mathbb{Z}}
	    \DeclareMathOperator{\Zp}{\mathbb{Z^{+}}}
        \DeclareMathOperator{\Zm}{\mathbb{Z^{-}}}
        \DeclareMathOperator{\N}{\mathbb{Z^{+}}}


	    % linear algebra stuff
		\DeclareMathOperator{\Ell}{\mathscr{L}}

		% stats stuff
		\DeclareMathOperator{\var}{\rm Var}
		\DeclareMathOperator{\sd}{\rm SD}
		\DeclareMathOperator{\SE}{\rm SE}

		% use pretty characters
		\DeclareMathOperator{\ep}{\varepsilon}
		\DeclareMathOperator{\ph}{\varphi}
	%% Levin's shorthand
		\newcommand{\E}{{\mathcal{E}}}
		\newcommand{\A}{{\mathcal{A}}}
		\newcommand{\B}{{\mathcal{B}}}
		\newcommand{\R}{{\mathbb{R}}}
		\newcommand{\X}{{\mathbf{X}}}
		\newcommand{\x}{{\mathbf{x}}}
		\newcommand{\M}{{\mathcal{M}}}
		\newcommand{\bvec}[1]{{\boldsymbol #1}}
		\newcommand{\bbeta}{\bvec{\beta}}
		\newcommand{\bX}{\bvec{X}}
		\newcommand{\bY}{\bvec{Y}}
		\newcommand{\ssreg}{{\rm SS}_{{\rm Reg}}}
		\newcommand{\ssr}{{\rm SS}_{{\rm Res}}}
		\newcommand{\sst}{{\rm SS}_{{\rm Tot}}}
\IfFileExists{upquote.sty}{\usepackage{upquote}}{}
\begin{document}
	\begin{abstract}
		I collaborated with Joel Bazzle, Torin Brown, Ashley Ordway and Seth Temple on this assignment.
	\end{abstract}

	\maketitle


	\section*{Problem 1} % (fold)
	\label{sec:problem_1}
		Let $x_{1} =(1,1,1,1,1,1)'$, $x_{2} = (3,−1,4,6,3,3)'$, $x_{3} = (7,3,2,0,3,3)'$, $x_{4} = (8,4,9,−5,4,4)'$, $Y = (4,36,44,12,16,8)'$, and $V = \Ell(x_{1},x_{2},x_{3},x_{4})$.
		Suppose we wish to test $H_{0} : \beta_{4} = 0, \beta_{2} = \beta_{3}$.
		\begin{enumerate}[(a)]
			\item Find two matrix $A$ such that $H_{0}$ is equivalent to $A\beta = 0$.
			\item Find $\hat{\beta}$, $\hat{Y} = X \hat{\beta}$, and $Z = A\hat{\beta}$ for one of your choices of $A$.
			\item Define $V_{0}$ so that $\theta := \mathbb{E}[Y|X] \in V_{0}$ if and only if $A\beta = 0$.
			Find $\hat{Y_{0}} = \Pi_{V_{0}}Y$, $Y - \hat{Y}$ and $\hat{Y_{1}} = \hat{Y} - \hat{Y_{0}}$.
			\item Determine $\rm SS_{Res} = ||Y - \hat{Y}||^{2}$, $\rm SS_{Res}(V_{0}) = ||Y - \hat{Y_{0}}||^{2}$, and the $F$-statistic.
			\item Verify that $||\hat{Y} - \hat{Y_{0}}||^{2} = Z'[A(X'X)^{-1}A']^{-1}Z$.
		\end{enumerate}
		\begin{proof}[Solution]






			\begin{enumerate}[(a)]
				\item Given that $H_{0} : \beta_{4} = 0, \beta_{2} = \beta_{3}$, and upon brushing up on linear algebra, we find the following matrices.
				Note that $A_{i}$ must have two rows because there are two linear constrains in $H_{0}$.
				We will use $A_{0}$ as our choice for $A_{i}$ for the rest of this problem.
					\begin{align*}
						A_{0} &= \begin{pmatrix}
							0 & 0 & 0 & 1 \\
							0 & 1 & -1 & 0
						\end{pmatrix} \\
						A_{1} &= \begin{pmatrix}
							0 & 0 & 0 & 1 \\
							0 & -1 & 1 & 0
						\end{pmatrix}
					\end{align*}
				\item We compute $\hat{\beta}$, $\hat{Y} = X \hat{\beta}$ and $Z = A\hat{\beta}$ in \verb|R|.
					\begin{align*}
					    \hat{\beta} &= (41, -3, -8, 3)', \\
					    \hat{Y} &= (\ensuremath{-1.07\times 10^{-14}}, 32, 40, 8, 20, 20)', \\
					    Z &= (3, 5)'.
					\end{align*}
				\item As before, most of the computational work is done in \verb|R|.
				Note that in order to define $V_{0}$ as requested, we combine $\beta_{2}$ and $\beta_{3}$ thus, $w = \beta_{2} + \beta_{3}$.
				We define $V_{0} = x_{1} + w$.
				Computing $\hat{Y_{0}}$, $Y - \hat{Y}$ and $\hat{Y_{1}}$ yields the following,
				    \begin{align*}
				        \hat{Y_{0}} &= (4, 36, 20, 20, 20, 20)', \\
				        Y - \hat{Y} &= (4, 4, 4, 4, -4, -12)', \rm and \\
				        \hat{Y_{1}} &= (-4, -4, 20, -12, \ensuremath{-3.55\times 10^{-14}}, \ensuremath{-3.55\times 10^{-14}})'.
				    \end{align*}
				\item Again, we compute the requested Residual Sums of Squares and $F$-statistic in \verb|R|.
				    \begin{align*}
				        \ssr &= 224, \\
				        \ssr(V_{0}) &= 800, \\
				        F &= \frac{800 - 224/2}{224/2}, \\
				        &= 2.57.
				    \end{align*}
				\item We compute that $||\hat{Y} - \hat{Y_{0}}||^{2} = 576$ and $Z'[A(X'X)^{-1}A']^{-1}Z = 576$.
				Observe that $576 = 576$.
			\end{enumerate}
		\end{proof}
	% section problem_1 (end)
	\section*{Problem 2} % (fold)
	\label{sec:problem_2}
		Consider the data in the dataset \verb|teengamb| in the package faraway:

			\verb|install.packages("faraway")|

			\verb|library(faraway)|

			\verb|data(teengamb)|

		The last line should bring up a description of the variables.
		Is there a difference between males and females as relates to gambling behavior?
		Fit any appropriate model(s) and carry out any appropriate test(s).
		\begin{proof}[Solution] We compare the following three models, and perform a $F$ test between each using an \verb|ANOVA| table.
		The first model considers all variables except sex, the second introduces sex as a variable, and the third model introduces sex as a dummy variable.
		We can see from the table that upon introducing sex as a descriptive variable we have significance, i.e., sex does have an impact on gambling behavior.
		Upon comparing sex generally with sex as a category among all other variables we again determine that this is significant, however less so than just considering sex alone.

		We should note however that there are several descriptive variables which haven't been measured/observed which are being absorbed by our error terms that are not independent from the variables which are controlling.
		These could be things such as parent's educational attainment, whether or not gambling is legal in the jurisdiction(s) from which we are performing observations, etc.
\begin{kframe}
\begin{alltt}
\hlstd{tg.lm0} \hlkwb{<-} \hlkwd{lm}\hlstd{(gamble}\hlopt{~}\hlstd{verbal}\hlopt{+}\hlstd{status}\hlopt{+}\hlstd{income,} \hlkwc{data} \hlstd{= teengamb)}
\hlstd{tg.lm1} \hlkwb{<-} \hlkwd{lm}\hlstd{(gamble}\hlopt{~}\hlstd{verbal}\hlopt{+}\hlstd{status}\hlopt{+}\hlstd{income}\hlopt{+}\hlstd{sex,} \hlkwc{data} \hlstd{= teengamb)}
\hlstd{tg.lm2} \hlkwb{<-} \hlkwd{lm}\hlstd{(gamble}\hlopt{~}\hlstd{verbal}\hlopt{+}\hlstd{status}\hlopt{+}\hlstd{income}\hlopt{+}\hlstd{sex}\hlopt{+}\hlstd{verbal}\hlopt{:}\hlstd{sex}
             \hlopt{+}\hlstd{status}\hlopt{:}\hlstd{sex}\hlopt{+}\hlstd{income}\hlopt{:}\hlstd{sex,} \hlkwc{data} \hlstd{= teengamb)}
\hlkwd{xtable}\hlstd{(}\hlkwd{anova}\hlstd{(tg.lm0, tg.lm1, tg.lm2))}
\end{alltt}
\end{kframe}% latex table generated in R 3.4.0 by xtable 1.8-2 package
% Fri May 19 17:53:20 2017
\begin{table}[ht]
\centering
\begin{tabular}{lrrrrrr}
  \hline
 & Res.Df & RSS & Df & Sum of Sq & F & Pr($>$F) \\ 
  \hline
1 & 43 & 25359.56 &  &  &  &  \\ 
  2 & 42 & 21623.77 & 1 & 3735.79 & 8.49 & 0.0059 \\ 
  3 & 39 & 17163.54 & 3 & 4460.23 & 3.38 & 0.0277 \\ 
   \hline
\end{tabular}
\end{table}

		\end{proof}
	% section problem_2 (end)

	\section*{Problem 3} % (fold)
	\label{sec:problem_3}
		Suppose that 11 plots of land are plotted with three varieties of corn.
		The following lists the yields for the three varieties:

		\begin{tabular}{ccc}
			I & II & III \\
			\hline
			52 & 64 & 53 \\
			56 & 57 & 55 \\
			60 & 62 & 58 \\
			56 & & 50
		\end{tabular}

		\begin{enumerate}[(a)]
			\item Test the hypothesis that the three varieties all have the same expected yield.
			\item Suppose that for the corn yield the true means were 70, 75, 95 and that $\sigma = 20$.
			Find the power of the $\alpha = 0.05$ level test for equal means.
			\item How large should $n_{0}$, the number of observations per treatment (number of plots per treatment) be in order to have power at least 0.90 for the parameters in (a)?
		\end{enumerate}
		\begin{proof}[Solution]



		\begin{enumerate}[(a)]
			\item In order to test $H_{0}: \mu_{1} = \mu_{2} = \mu_{3}$ we perform an $F$ test, in this case we refer to $\rm RSS_{Full}$ as the within group variability (SSW), and $\rm RSS_{Small}$ as the between group variability (SSB).
			We find that $\mu_{1} = 56$, $\mu_{2} = 61$, and $\mu_{3} = 54$, thus the grand mean is $\mu_{\rm G} = n^{-1}\sum_{i=1}^{3} \mu_{i} = 57$.
			With these quantities we can compute the SSB as follows,\footnote{More detailed computations are shown in the R chunks.}
				\begin{align*}
					\rm SSB &= \sum\limits_{i=1}^{3} n_{i} \left( \bar{Y}_{i,.} - \bar{Y}_{.,.} \right)^{2}, \\
					&= 88.
				\end{align*}
			For SSW we sum over the variances of each column, weighted by their size,
				\begin{align*}
					\rm SSW &= \sum\limits_{i=1}^{3}\sum\limits_{j=1}^{n_{i}} \left( Y_{i,j} - \bar{Y}_{i,.} \right)^{2}, \\
					&= 127.
				\end{align*}
			We compute the $F$-statistic as follows,
				\begin{align*}
					F &= \frac{SSB/df_{B}}{SSW/df_{W}}, \\
					&= \frac{88/2}{127/8}, \\
					&= 2.77.
				\end{align*}
			We find that the p-value is 0.12, given that this is greater than most reasonable level $\alpha$'s, we fail to reject $H_{0}$
			\item In order to compute the power for $H_{1}: \mu_{1} = 70, \mu_{2} = 75$ and $\mu_{3} = 95$, with $\sigma^{2} = 400$ we must find the grand mean for these supposed $\mu{i}$'s, and with that compute the Corrected Sum of Squares (CSS).
			We do these computations in \verb|R| and find that the new grand mean $\mu_{G_{1}} = \sum_{i=1}^{3} \frac{n_{i}}{11} \bar{Y}_{i,.} = 80.45$.
			With this new grand mean we find that CSS = 1372.73.
			When computing the power we must compute our non-centrality parameter $\delta$, for one-way analysis of variance this is simply $\delta = \rm CSS/\sigma^{2} = 3.43$.
			If we suppose $\alpha = 0.05$, we compute that the power is $0.26$, which is too low for us to say that our conclusion in (a) is well-founded.
			\item Here I use Seth's idea to use a for loop to try new sample sizes starting at $n_{i} = 20$ down to $n_{i} = 4$ (our largest sample), with the loop terminating when the power dips below our desired value of 0.9.
			Given this method we determine that the least sample size for which we have a power of at least 0.9 is $n_{i} = 16$.
		\end{enumerate}
		\end{proof}
	% section problem_3 (end)
\end{document}
