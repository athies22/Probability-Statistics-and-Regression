\documentclass{amsart}\usepackage[]{graphicx}\usepackage[]{color}
%% maxwidth is the original width if it is less than linewidth
%% otherwise use linewidth (to make sure the graphics do not exceed the margin)
\makeatletter
\def\maxwidth{ %
  \ifdim\Gin@nat@width>\linewidth
    \linewidth
  \else
    \Gin@nat@width
  \fi
}
\makeatother

\definecolor{fgcolor}{rgb}{0.345, 0.345, 0.345}
\newcommand{\hlnum}[1]{\textcolor[rgb]{0.686,0.059,0.569}{#1}}%
\newcommand{\hlstr}[1]{\textcolor[rgb]{0.192,0.494,0.8}{#1}}%
\newcommand{\hlcom}[1]{\textcolor[rgb]{0.678,0.584,0.686}{\textit{#1}}}%
\newcommand{\hlopt}[1]{\textcolor[rgb]{0,0,0}{#1}}%
\newcommand{\hlstd}[1]{\textcolor[rgb]{0.345,0.345,0.345}{#1}}%
\newcommand{\hlkwa}[1]{\textcolor[rgb]{0.161,0.373,0.58}{\textbf{#1}}}%
\newcommand{\hlkwb}[1]{\textcolor[rgb]{0.69,0.353,0.396}{#1}}%
\newcommand{\hlkwc}[1]{\textcolor[rgb]{0.333,0.667,0.333}{#1}}%
\newcommand{\hlkwd}[1]{\textcolor[rgb]{0.737,0.353,0.396}{\textbf{#1}}}%
\let\hlipl\hlkwb

\usepackage{framed}
\makeatletter
\newenvironment{kframe}{%
 \def\at@end@of@kframe{}%
 \ifinner\ifhmode%
  \def\at@end@of@kframe{\end{minipage}}%
  \begin{minipage}{\columnwidth}%
 \fi\fi%
 \def\FrameCommand##1{\hskip\@totalleftmargin \hskip-\fboxsep
 \colorbox{shadecolor}{##1}\hskip-\fboxsep
     % There is no \\@totalrightmargin, so:
     \hskip-\linewidth \hskip-\@totalleftmargin \hskip\columnwidth}%
 \MakeFramed {\advance\hsize-\width
   \@totalleftmargin\z@ \linewidth\hsize
   \@setminipage}}%
 {\par\unskip\endMakeFramed%
 \at@end@of@kframe}
\makeatother

\definecolor{shadecolor}{rgb}{.97, .97, .97}
\definecolor{messagecolor}{rgb}{0, 0, 0}
\definecolor{warningcolor}{rgb}{1, 0, 1}
\definecolor{errorcolor}{rgb}{1, 0, 0}
\newenvironment{knitrout}{}{} % an empty environment to be redefined in TeX

\usepackage{alltt}
\usepackage{enumerate,amssymb,graphicx,verbatim}
\usepackage{fourier}
\usepackage[T1]{fontenc}
\usepackage[colorlinks=true, pdfstartview=FitV, linkcolor=blue,
 citecolor=blue, urlcolor=red]{hyperref}

\newcommand{\E}{{\mathbb E}}
\newcommand{\A}{{\mathcal A}}
\newcommand{\B}{{\mathcal B}}
\newcommand{\R}{{\mathbb R}}
\newcommand{\st}{\;:\;}
\newcommand{\M}{{\mathcal M}}
\newcommand{\var}{{\rm Var}}
\newcommand{\ep}{\varepsilon}
\newcommand{\sd}{{\rm SD}}
\newcommand{\bvec}[1]{{\boldsymbol #1}}
\newcommand{\X}{\bvec{X}}
\newcommand{\x}{\bvec{x}}
\newcommand{\Y}{\bvec{Y}}
\newcommand{\bbeta}{\bvec{\beta}}
\newcommand{\Lin}{{\mathcal L}}
\newcommand{\rss}{{\rm RSS}}


\newenvironment{enumeratei}{\begin{enumerate}[\upshape (i)]}
                           {\end{enumerate}}
\newenvironment{enumeratea}{\begin{enumerate}[\upshape (a)]}
                           {\end{enumerate}}

\newenvironment{enumeraten}{\begin{enumerate}[\upshape 1.]}
                           {\end{enumerate}}


\newenvironment{Problem}[1]
 {
   \medskip
   \noindent \textbf{\hypertarget{X:#1}{Problem #1.}}
 }
 {
   %\hyperlink{Sol:#1}{Solution}
 }

\newenvironment{ProblemSol}[1]{
  \begin{proof}[\hypertarget{Sol:#1}{Solution to Problem }\hyperlink{X:#1}{#1}]
}
{
\end{proof}
}
\newtheorem{theorem}{Theorem}
\theoremstyle{definition}
\newtheorem{exercise}{Exercise}
\newtheorem{question}{Question}
\newcommand{\LinkSol}[1]{                          %%
\hfill \hyperlink{Sol:#1}{\framebox{Solution $\checkmark$}}}

\title{Calculating Power of $F$-test}
\IfFileExists{upquote.sty}{\usepackage{upquote}}{}
\begin{document}

\maketitle


\section{Some Theory}

We have our usual linear model:
\[
\bvec{Y} = \X \bbeta + \bvec{\ep}
\]
where it is assumed that $\bvec{\ep}$ is independent of
$\bvec{X}$ and $\bvec{\ep} \sim N(\bvec{0}, \sigma^2I)$.

We want to consider testing
\begin{equation} \label{Eq:Hy}
H_0 : \beta_{r+1} = \cdots = \beta_k = 0 \,.
\end{equation}
Let
\begin{align*}
  V_0 & = \Lin(\x_1,\ldots,\x_r) \\
  V_1 & = \Lin(\x_1,\ldots,\x_k) \,.
\end{align*}
and if $\widehat{\bvec{Y}^{(i)}} = \Pi_{V_i} \bvec{Y}$,
then
\[
\rss_i = \sum_{j=1}^n (Y_j - \widehat{Y_j^{(i)}})^2
= \| \bvec{Y} - \widehat{\bvec{Y}^{(i)}} \|^2 \,.
\]
We know that under $H_0$,
\[
F =
\frac{\|\widehat{\bvec{Y}^{(0)}} - \widehat{\bvec{Y}^{(1)}}\|^2/(k-r)}{\|\bvec{Y}-\widehat{\bvec{Y}^{(1)}}\|^2/(n-k)}
= \frac{(\rss_0 - \rss_1)/(k-r)}{\rss_1/(n-k)}
\]
has an $F$-distribution with $k-r,n-k$ degrees of freedom.

To carry out this test in $R$, fit two models and use
\verb|anova| to compute the $F$-statistic.  See
Table \ref{Tab:F} for an example.

\begin{kframe}
\begin{alltt}
\hlkwd{library}\hlstd{(xtable)}
\hlstd{boston} \hlkwb{=} \hlkwd{read.table}\hlstd{(}\hlkwd{url}\hlstd{(}\hlstr{"http://pages.uoregon.edu/dlevin/DATA/BostonB.txt"}\hlstd{),}
                    \hlkwc{header}\hlstd{=T)}
\hlstd{g1} \hlkwb{=} \hlkwd{lm}\hlstd{(medv}\hlopt{~}\hlstd{.,}\hlkwc{data}\hlstd{=boston)}
\hlstd{g2} \hlkwb{=} \hlkwd{lm}\hlstd{(medv}\hlopt{~}\hlstd{.}\hlopt{-}\hlstd{age}\hlopt{-}\hlstd{indus,}\hlkwc{data}\hlstd{=boston)}
\hlkwd{xtable}\hlstd{(}\hlkwd{anova}\hlstd{(g2,g1),}\hlkwc{caption}\hlstd{=}\hlstr{"F test on coefficients of age and industry"}\hlstd{,}
       \hlkwc{label}\hlstd{=}\hlstr{"Tab:F"}\hlstd{)}
\end{alltt}
\end{kframe}% latex table generated in R 3.4.0 by xtable 1.8-2 package
% Sun May  7 16:14:39 2017
\begin{table}[ht]
\centering
\begin{tabular}{lrrrrrr}
  \hline
 & Res.Df & RSS & Df & Sum of Sq & F & Pr($>$F) \\ 
  \hline
1 & 494 & 11081.36 &  &  &  &  \\ 
  2 & 492 & 11078.78 & 2 & 2.58 & 0.06 & 0.9443 \\ 
   \hline
\end{tabular}
\caption{F test on coefficients of age and industry} 
\label{Tab:F}
\end{table}



If we want to compute probabilities for $F$ without the
assumption of $H_0$ (e.g., to compute the power), we need:

\begin{theorem}
  The distribution of $F$ is a \emph{non-central}
  $F$-distribution with degrees of freedom $k-r$ and
  $n-k$ and with \emph{non-centrality parameter}
  \[
  \delta = \frac{ \| \Pi_{V_1 \cap V_0^\perp} \bvec{\theta} \|^2}{\sigma^2} \,,
  \]
  where $\bvec{\theta} = \E[\bvec{Y}\mid \X]$.
\end{theorem}

Note that
\[
\Pi_{V_1 \cap V_0^\perp} \bvec{\theta}
= \sum_{j=r+1}^k \beta_j (\x_j - \Pi_{V_0} \x_k)
= \sum_{j=j+1}^k \beta_k \x_j^\perp,
\]
where $\x_j^\perp = \x_j - \Pi_{V_0} \x_k$.

Thus,
\[
\delta =
\frac{1}{\sigma^2}\left[\sum_{j=r+1}^k \beta_j^2 \| \x_j^\perp \|^2
  + 2\sum_{r+1 \leq i < j \leq k} \beta_i\beta_j \langle \x_i^\perp, \x_j^\perp \rangle \right]\,.
\]

\begin{question}
  The parameter $\delta$ is ``unitless'', i.e. it does not
  depend on the units of any of the variables.  Why?
\end{question}

To calculate $\delta$, we need to calculate $\x_j^\perp$.
There are several ways to do this.   Note that
$\x_j^\perp$ is the vector of residuals when performing OLS
of $\x_j$ against $\x_1, \ldots, \x_r$.  (Why?)

Suppose $\beta_{{\rm indus}} = 0.05$ and
$\beta_{{\rm age}} = 0.001$.  Let us find $\delta$:

\begin{knitrout}
\definecolor{shadecolor}{rgb}{0.969, 0.969, 0.969}\color{fgcolor}\begin{kframe}
\begin{alltt}
\hlstd{ageperp} \hlkwb{=} \hlkwd{residuals}\hlstd{(}\hlkwd{lm}\hlstd{(age}\hlopt{~}\hlstd{.}\hlopt{-}\hlstd{medv}\hlopt{-}\hlstd{indus,}\hlkwc{data}\hlstd{=boston))}
\hlstd{indusperp} \hlkwb{=} \hlkwd{residuals}\hlstd{(}\hlkwd{lm}\hlstd{(indus}\hlopt{~}\hlstd{.}\hlopt{-}\hlstd{medv}\hlopt{-}\hlstd{age,}\hlkwc{data}\hlstd{=boston))}
\hlstd{de} \hlkwb{=} \hlstd{(}\hlnum{0.05}\hlopt{^}\hlnum{2}\hlopt{*}\hlkwd{sum}\hlstd{(ageperp}\hlopt{^}\hlnum{2}\hlstd{)} \hlopt{+} \hlnum{0.001}\hlopt{^}\hlnum{2}\hlopt{*}\hlkwd{sum}\hlstd{(indusperp}\hlopt{^}\hlnum{2}\hlstd{)}
      \hlopt{+} \hlnum{2}\hlopt{*}\hlnum{0.001}\hlopt{*}\hlnum{0.05}\hlopt{*}\hlkwd{sum}\hlstd{(ageperp}\hlopt{*}\hlstd{indusperp))}\hlopt{/}\hlkwd{summary}\hlstd{(g1)}\hlopt{$}\hlstd{sigma}\hlopt{^}\hlnum{2}
\end{alltt}
\end{kframe}
\end{knitrout}

Now we find the cut-off for the $F$-test, say with significance level $0.05$:

\begin{knitrout}
\definecolor{shadecolor}{rgb}{0.969, 0.969, 0.969}\color{fgcolor}\begin{kframe}
\begin{alltt}
\hlstd{fstar} \hlkwb{=} \hlkwd{qf}\hlstd{(}\hlnum{0.95}\hlstd{,}\hlnum{2}\hlstd{,}\hlnum{492}\hlstd{)}
\end{alltt}
\end{kframe}
\end{knitrout}

To find the power:
\begin{knitrout}
\definecolor{shadecolor}{rgb}{0.969, 0.969, 0.969}\color{fgcolor}\begin{kframe}
\begin{alltt}
\hlnum{1} \hlopt{-} \hlkwd{pf}\hlstd{(fstar,}\hlnum{2}\hlstd{,}\hlnum{492}\hlstd{,}\hlkwc{ncp} \hlstd{= de)}
\end{alltt}
\begin{verbatim}
## [1] 0.9322011
\end{verbatim}
\end{kframe}
\end{knitrout}
\begin{question}
The design matrix $\bvec{X}$ is said to have
\emph{collinearity} if there are \emph{near} linear relationships among the columns.  Why is the power of the $F$-test
limited when the variables $\x_j$ for $j > r$ are nearly
collinear with the variables $\x_j$ for $j \leq r$.
\end{question}

\section{An example}

This example concerns the \verb|crime| dataset, available
via
\begin{knitrout}
\definecolor{shadecolor}{rgb}{0.969, 0.969, 0.969}\color{fgcolor}\begin{kframe}
\begin{alltt}
\hlstd{crime} \hlkwb{=} \hlkwd{read.table}\hlstd{(}\hlkwd{url}\hlstd{(}\hlstr{"http://pages.uoregon.edu/dlevin/DATA/crime.txt"}\hlstd{),}\hlkwc{header}\hlstd{=T)}
\end{alltt}
\end{kframe}
\end{knitrout}
A description of this data set is at

\url{http://pages.uoregon.edu/dlevin/DATA/USCrimeDatafile.html}

Here, the mortality rate \verb|R| is modelled as a function of the other variables.


First, use OLS to fit a model including all the variables.

\begin{question}
Run a \verb|summary| of the OLS model.  Note the most of the coefficients are ``not significant''.
Test the hypothesis that none of the ``not significant'' coefficients are non-zero.  What do you find?  Discuss.
\end{question}¯

\begin{question}
In the first \verb|summary| of the full OLS model, the coeffiecients of \verb|Ex0| and \verb|Ex1| have different signs.
Looking at the description of the variables, does this make sense?  How would you explain this?
\end{question}
\begin{question}
  In the above question, running the \verb|summary| command performed several tests.  What were these tests?  Afterwards, the question asked you to perform another test, \emph{based on the results} given in \verb|summary|.  Does this matter in interpreting the results of the second test?
\end{question}

Now, consider the test that the
coefficients of both \verb|Ex1| and \verb|U1| are zero.

\begin{question}
What is the power of this test when $\beta_{{\rm Ex1}} = -1$
and $\beta_{{\rm U1}} = -1$.
\end{question}
First, carry out the $F$-test that these two coefficients are zero.


\end{document}
