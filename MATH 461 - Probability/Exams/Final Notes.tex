\documentclass{amsart}
\usepackage{amsmath}
\usepackage{amssymb}
\usepackage{amsthm}
\usepackage{enumerate}
\DeclareMathOperator{\Cov}{\text{Cov}}
\DeclareMathOperator{\Var}{\text{Var}}
\DeclareMathOperator{\bin}{\text{Binomial}}
\DeclareMathOperator{\bern}{\text{Bernoulli}}
\DeclareMathOperator{\fish}{\text{Poisson}}
\DeclareMathOperator{\uni}{\text{Uniform}}
\DeclareMathOperator{\Norm}{\text{Norm}}
\begin{document}
	\author{Alex Thies}
	\title{Final Exam Notes \\ Math 461 - Fall 2016}
	\email{athies@uoregon.edu}

	\maketitle
	
	\section{Discrete Random Variables}
		\subsection{Summary}
			\subsubsection{Probability Mass Function}
			
			It is important to keep in mind that with discrete random variables we compute the probability that a random variable is equal to a certain value, i.e., for $X$ discrete, we compute $p_{X}(x) = P(X = x)$, this is in contrast to the continuous case. Therefore, when asked to compute something like $P(X > n)$, we must actually compute $1 - \sum_{i=0}^{n-1}p_{X}(i)$; or for $P(X < n)$, we actually compute $\sum_{i=0}^{n-1} p_{X}(i)$.
			
				\subsubsection{Expectation} 
				The expected value of a discrete random variable is computed with the following formula,
					$$
						E(X) = \sum_{x \in X} x p_{X}(x)
					$$				
					
					When asked to compute a composition of $X$, we need only manipulate the \textit{weighted} $x$, rather than the argument of the probability mass function, e.g.,
					\begin{align*}
						E(X) &=  \sum_{x} x p(x); \\
						E(X^{2}) &=  \sum_{x} x^{2} p(x); \\
						E(1/X) &=  \sum_{x} x^{-1} p(x); \\
						E(\log(X)) &= \sum_{x} \log(x) p(x).
					\end{align*}
					
				\subsubsection{Variance}
				The variance of a discrete random variable is the expected value of the squared difference between the expectation and the result, i.e.,
					\begin{align*}
						\Var(X) &= E\left[(X-\mu)^{2}\right] \\
						&= \sum_{x \in X} (x - \mu)^{2} p_{X}(x) \\
						&= \sum_{x} x^{2} p(x) - 2\mu\sum_{x} x p(x) + \mu^{2} \sum_{x} p(x) \\
						&= E(X^{2}) - 2\mu^{2} + \mu^{2} \\
						&= E(X^{2}) - \mu^{2}
					\end{align*}
					\begin{align*}
						\Var(X) &= E(X^{2}) - \left[E(X)\right]^{2} \\
						&= E(X^{2}) - E^{2}(X)
					\end{align*}
				Furthermore, note that 
					$$
						\Var(aX + b) = a^{2}\Var(X)
					$$
		
		\subsection{Binomial Random Variables}
		A Binomial Random Variable is the classic case of an experiment in which there is a binary value assigned to success of an event which is attempted $n$ times, with the probability of success being equal to some $p \in [0,1]$. The classic case is tossing a coin $n$ times, with probability of landing on heads equalling $p$; these values serve as the parameters of the random variable, i.e., $X \sim \bin(n,p)$.
			\subsubsection{Probability Mass Function}
			For $X \sim \bin(n,p)$,
				$$
					p_{X}(x) = P(X = x) = \binom{n}{x} p^{x} (1-p)^{n-x}
				$$
				
			\subsubsection{Expectation of Binomial}
			For $X \sim \bin(n,p)$,
				$$
					E(X) = 	np
				$$
				
			\subsubsection{Variance of Binomial}
			For $X \sim \bin(n,p)$,
				$$
					\Var(X) = np(1-p)
				$$
		
		\subsection{Poisson Random Variables}
		A Poisson Random Variable approximates a Binomial Random Variable given large $n$ and small $p$, the product, $\lambda$, of these values serve as the parameters of the random variable, i.e., $X \sim \fish(\lambda)$.
				\subsubsection{Probability Mass Function}
				For $X \sim \fish(\lambda)$, 
					$$
						p_{X}(x) = P(X=x) = \frac{\lambda^{x}e^{-\lambda}}{x!}
					$$
					
				\subsubsection{Expectation, and Variance of Poisson} 
				For $X \sim \fish(\lambda)$,
					$$
						E(X) = \Var(X) = \lambda.
					$$
					
			\subsection{Bernoulli Random Variables}
			A Bernoulli Random Variable is the special case where a Binomial Random Variable has one trial, e.g., coin tosses, independent events with binary outcome; thus for probability of success = $p$, 
				$$
					X \sim \bern(p) \Leftrightarrow X \sim \bin(1,p).
				$$
			
				\subsubsection{Probability Mass Function}
				For $X \sim \bern(p)$,
					\begin{align*}
						p(0) &= P(X = 0) = 1 - p, \\
						p(1) &= P(X = 1) = p.
					\end{align*}
					
				\subsubsection{Expectation}
				For $X \sim \bern(p)$,
					$$
						E(X) = p.
					$$
				
		\section{Continuous Random Variables}
			\subsection{Summary}
				\subsubsection{Probability Density Function (PDF)}
				
				\subsubsection{Cumulative Distribution Function (CDF)}
			
				\subsubsection{Expectation} 
				The expected value of a continuous random variable is computed with the following formula,
					$$
						E(X) = \int_{-\infty}^{\infty} x f_{X}(x) \ dx
					$$
				Note that this is fairly similar to the expectation of a discrete random variable, but in the language of continous functions.			
					
				\subsubsection{Variance}
			\subsection{Normal Random Variables}
				\subsubsection{PDF and CDF}
					
				\subsubsection{Expecatation}
					
				\subsubsection{Variance}
				
				\subsubsection{Covariance}
				
				\subsubsection{Normal Approximation of Binomial Random Variable}
				
			\subsection{Standardized Normal}
				\subsubsection{PDF and CDF}
					
				\subsubsection{Expecatation}
					
				\subsubsection{Variance}
				
				\subsubsection{Covariance}
				
			\subsection{Exponential Random Variables}
				\subsubsection{PDF and CDF}
				For $X \sim \exp(\lambda)$,
					$$
						f_{X}(x) = \begin{cases} \lambda e^{-\lambda x} & x \geq 0 \\ 0 & o/w \end{cases}
					$$
					$$
						F_{X}(x) = P(X \leq x) = \int_{0}^{x} \lambda e^{-\lambda t} \ dt
					$$
					
				\subsubsection{Expecatation}
					
				\subsubsection{Variance}
				
				\subsubsection{Covariance}
				
		\section{Jointly Distributed Random Variables}

\end{document}